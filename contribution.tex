\documentclass[]{article}
\usepackage{lmodern}
\usepackage{amssymb,amsmath}
\usepackage{ifxetex,ifluatex}
\usepackage{fixltx2e} % provides \textsubscript
\ifnum 0\ifxetex 1\fi\ifluatex 1\fi=0 % if pdftex
  \usepackage[T1]{fontenc}
  \usepackage[utf8]{inputenc}
\else % if luatex or xelatex
  \ifxetex
    \usepackage{mathspec}
  \else
    \usepackage{fontspec}
  \fi
  \defaultfontfeatures{Ligatures=TeX,Scale=MatchLowercase}
    \setmainfont[]{Times New Roman}
\fi
% use upquote if available, for straight quotes in verbatim environments
\IfFileExists{upquote.sty}{\usepackage{upquote}}{}
% use microtype if available
\IfFileExists{microtype.sty}{%
\usepackage{microtype}
\UseMicrotypeSet[protrusion]{basicmath} % disable protrusion for tt fonts
}{}
\usepackage[margin=1in]{geometry}
\usepackage{hyperref}
\hypersetup{unicode=true,
            pdftitle={Contributing},
            pdfborder={0 0 0},
            breaklinks=true}
\urlstyle{same}  % don't use monospace font for urls
\usepackage{graphicx,grffile}
\makeatletter
\def\maxwidth{\ifdim\Gin@nat@width>\linewidth\linewidth\else\Gin@nat@width\fi}
\def\maxheight{\ifdim\Gin@nat@height>\textheight\textheight\else\Gin@nat@height\fi}
\makeatother
% Scale images if necessary, so that they will not overflow the page
% margins by default, and it is still possible to overwrite the defaults
% using explicit options in \includegraphics[width, height, ...]{}
\setkeys{Gin}{width=\maxwidth,height=\maxheight,keepaspectratio}
\IfFileExists{parskip.sty}{%
\usepackage{parskip}
}{% else
\setlength{\parindent}{0pt}
\setlength{\parskip}{6pt plus 2pt minus 1pt}
}
\setlength{\emergencystretch}{3em}  % prevent overfull lines
\providecommand{\tightlist}{%
  \setlength{\itemsep}{0pt}\setlength{\parskip}{0pt}}
\setcounter{secnumdepth}{0}
% Redefines (sub)paragraphs to behave more like sections
\ifx\paragraph\undefined\else
\let\oldparagraph\paragraph
\renewcommand{\paragraph}[1]{\oldparagraph{#1}\mbox{}}
\fi
\ifx\subparagraph\undefined\else
\let\oldsubparagraph\subparagraph
\renewcommand{\subparagraph}[1]{\oldsubparagraph{#1}\mbox{}}
\fi

%%% Use protect on footnotes to avoid problems with footnotes in titles
\let\rmarkdownfootnote\footnote%
\def\footnote{\protect\rmarkdownfootnote}

%%% Change title format to be more compact
\usepackage{titling}

% Create subtitle command for use in maketitle
\providecommand{\subtitle}[1]{
  \posttitle{
    \begin{center}\large#1\end{center}
    }
}

\setlength{\droptitle}{-2em}

  \title{Contributing}
    \pretitle{\vspace{\droptitle}\centering\huge}
  \posttitle{\par}
    \author{}
    \preauthor{}\postauthor{}
    \date{}
    \predate{}\postdate{}
  

\begin{document}
\maketitle

\textbf{Thank you for considering to help out!}

QDECR started as a side project to make QDEC and mri\_glmfit more
accessible to members of our department. The desire for additional
features kept growing, and so we decided to reprogram everything in R.
Over time we have developed QDECR to fit our specific needs, but those
may differ from what you - the user - may need. We therefore decided to
make QDECR accessible early on, so that it can become a community-driven
project.

A quick overview of what you will find in this document:

\begin{itemize}
\tightlist
\item
  \protect\hyperlink{before-contributing}{Before Contributing}

  \begin{itemize}
  \tightlist
  \item
    \protect\hyperlink{code-of-conduct}{Code Of Conduct}
  \item
    \protect\hyperlink{communication}{Communication}
  \end{itemize}
\item
  \protect\hyperlink{ways-to-contribute}{Ways To Contribute}

  \begin{itemize}
  \tightlist
  \item
    \protect\hyperlink{finding-and-reporting-bugs}{Finding And Reporting
    Bugs}
  \item
    \protect\hyperlink{modifying-or-adding-code}{Modifying Or Adding
    Code}
  \item
    \protect\hyperlink{suggesting-ideas}{Suggesting Ideas}
  \item
    \protect\hyperlink{developing-educationalux5cux2520resources}{Developing
    Educational Resources}
  \end{itemize}
\item
  \protect\hyperlink{how-to-contribute}{How To Contribute}
\item
  \protect\hyperlink{list-of-features-to-implement}{List Of Features To
  Implement}

  \begin{itemize}
  \tightlist
  \item
    \protect\hyperlink{lower-resource-usage}{Lower Resource Usage}
  \item
    \protect\hyperlink{permutation-testing}{Permutation Testing}
  \item
    \protect\hyperlink{mixed-models}{Mixed Models}
  \item
    \protect\hyperlink{input-agnosticism}{Input Agnosticism}
  \end{itemize}
\end{itemize}

\hypertarget{before-contributing}{%
\subsection{Before Contributing}\label{before-contributing}}

\hypertarget{code-of-conduct}{%
\subsubsection{Code Of Conduct}\label{code-of-conduct}}

Working on open source projects is extremely fulfilling, but it does
require everyone to work together. As such we work with a
\href{code-of-conduct.html}{code of conduct (found here)} for the QDECR
project.

\hypertarget{communication}{%
\subsubsection{Communication}\label{communication}}

I created a \href{https://gitter.im}{Gitter} community for QDECR that
can be found \href{https://gitter.im/QDECR/}{here}:

\begin{itemize}
\tightlist
\item
  questions: For questions about the package
\item
  dev: For communication related to contributions (AKA this page!)
\item
  QDECR: A general room for anything
\end{itemize}

\hypertarget{ways-to-contribute}{%
\subsection{Ways To Contribute}\label{ways-to-contribute}}

\hypertarget{finding-and-reporting-bugs}{%
\subsubsection{Finding And Reporting
Bugs}\label{finding-and-reporting-bugs}}

You may notice issues with QDECR that seem unintentional. These can be
reported on \href{https://github.com/slamballais/QDECR/issues}{the
issues page on Github}.

\hypertarget{modifying-or-adding-code}{%
\subsubsection{Modifying Or Adding
Code}\label{modifying-or-adding-code}}

Users are also free to modify and add code through pull requests.

\hypertarget{suggesting-ideas}{%
\subsubsection{Suggesting Ideas}\label{suggesting-ideas}}

QDECR is a framework that can be extended into many directions. A lot of
users have already reported ideas, which we are often happy to
implement. However, due to the limited time the main developers we
cannot implement all suggestions. We aim to implement features that:

\begin{itemize}
\tightlist
\item
  Help QDECR to become more canonical
\item
  Address problems many users have
\item
  Open QDECR to new fields of study
\end{itemize}

Users can suggest ideas in the \href{https://gitter.im/QDECR/}{Gitter
community}. A lot of priority projects can be found below, under
\protect\hyperlink{list-of-features-to-implement}{List Of Features To
Implement}.

\hypertarget{developing-educational-resources}{%
\subsubsection{Developing Educational
Resources}\label{developing-educational-resources}}

We originally developed QDECR to help out members of our department who
were not familiar with Bash, design matrices, etc. Making QDECR
accessible is therefore one of our main goals. We therefore also welcome
all resources that will help explain the package to others. Several
possibilities:

\begin{itemize}
\tightlist
\item
  Expanding the tutorial
  (\href{https://github.com/slamballais/QDECR/tree/gh-pages}{in the
  gh-pages branch})
\item
  Writing blog posts with quick tutorials
\item
  Help with writing the documentation
  (\href{https://github.com/slamballais/QDECR/tree/dev}{in the dev
  branch})
\end{itemize}

\hypertarget{how-to-contribute}{%
\subsection{How To Contribute}\label{how-to-contribute}}

QDECR generally works with three branches:

\begin{itemize}
\tightlist
\item
  \href{https://github.com/slamballais/QDECR/tree/master}{\textbf{master}}:
  The stable version of the package that normal users work with
\item
  \href{https://github.com/slamballais/QDECR/tree/dev}{\textbf{dev}}:
  The development version for modification and pull requests
\item
  \href{https://github.com/slamballais/QDECR/tree/gh-pages}{\textbf{gh-pages}}:
  The code for the website
\end{itemize}

The general workflow for contributing:

\begin{itemize}
\tightlist
\item
  Consider where you want to contribute:

  \begin{itemize}
  \tightlist
  \item
    \textbf{QDECR package}: Please branch off
    \href{https://github.com/slamballais/QDECR/tree/dev}{the dev
    branch}. This helps us control the release of minor and major
    changes.
  \item
    \textbf{website}: Please branch off the
    \href{https://github.com/slamballais/QDECR/tree/gh-pages}{gh-pages
    branch}.
  \end{itemize}
\item
  Look through the issues and see if any match what you want to do. If
  so, leave a comment there with what you want to do.
\item
  If your issue is unique, then open a new issue.
\item
  Make sure that any pull request is small and self-contained. To smooth
  out the process, try breaking up big changes into smaller ones. This
  helps with keeping an overview of what is ready to be merged and what
  is not.
\item
  Feedback on the pull request can take some time and the request may
  need some iterations of modification. Still, we aim to incorporate all
  modifications that improve the QDECR project.
\end{itemize}

\hypertarget{list-of-features-to-implement}{%
\subsection{List Of Features To
Implement}\label{list-of-features-to-implement}}

At OHBM 2019 we had both a poster and a software demo for QDECR. During
those sessions, many of you had great suggestions on how to improve the
project. Below we have outlined some of those suggestions that have a
high priority for us.

\hypertarget{lower-resource-usage}{%
\subsubsection{Lower Resource Usage}\label{lower-resource-usage}}

QDECR has always been a typical programming project: first get it
working, then get it working smoothly. We have had several iterations of
the vertex-wise code and of storing all the data, and we believe several
more iterations are needed to optimize speed and reduce RAM burden. We
have some solutions that we will try to implement soon, but any
suggestions here are welcome.

\hypertarget{permutation-testing}{%
\subsubsection{Permutation Testing}\label{permutation-testing}}

QDECR currently only allows for one type of multiple testing correction.
Ideally, we would implement full permutation testing, but this is not
feasible. Luckily,
\href{https://www.ncbi.nlm.nih.gov/pmc/articles/PMC5035139/}{several
solutions exist to approximate full permutation testing}. We aim to
implement at least one of these approximations to improve statistical
inference.

\hypertarget{mixed-models}{%
\subsubsection{Mixed Models}\label{mixed-models}}

There is a large demand for mixed modeling, for example for longitudinal
studies.

\hypertarget{input-agnosticism}{%
\subsubsection{Input Agnosticism}\label{input-agnosticism}}

The package is currently written to work with Freesurfer output.
However, we have gotten multiple requests to allow for input from other
software (CIVET, ANTs) and even other modalities (fMRI). We will likely
implement a flexible module that calls functions for reading in the
input. Ideally, users should be able to write their own i/o functions
that can be used without our interference.


\end{document}
